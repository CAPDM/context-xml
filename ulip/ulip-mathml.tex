%D \module
%D   [       file=x-mathml,
%D        version=2008.05.29, (evolved from pre 2000 code)
%D          title=\CONTEXT\ XML Modules,
%D       subtitle=\MATHML,
%D         author=Hans Hagen,
%D           date=\currentdate,
%D      copyright={PRAGMA ADE \& \CONTEXT\ Development Team}]
%C
%C This module is part of the \CONTEXT\ macro||package and is
%C therefore copyrighted by \PRAGMA. See mreadme.pdf for
%C details.

% \xmlfilter{#1}{/*/name()} -> \xmltag

% This implementation looks like a hack ... this is because we deal with all weird
% cases we ran into, including abuse that was supposed to render ok (even if it
% didn't in other renderers) .. it was simply expected to work that way. It is also
% the product of stepwise adaptation to what we run into. And that in turn is often
% determined by the tools that were used to produce the \XML. Even worse is the
% cycling between coding that happens in projects: from content\ \MATHML\ to
% OpenMath to asciimath to presentation \MATHML\ to \unknown\ which all depends on
% what tools are available to render in a browser (where first \MATHML\ was hardly
% supported, then became accepted, then was taken over by mathjax, then merged with
% other input methods and who knopws what comes next in these days where technologies
% are shortlived and the next best thing is already waiting around the corner.
%
% So, consider this module to be under constant construction and clean up. We still
% use a funny mix of xml, tex and lua. I could rewrite the lot but it also shows
% how context evolves. I might end up with a lua-only implementation some day, but
% I must find a real good reason to spend time on it as so far it never paid back.
% As far as I can tell, publishers have never shown much (real) interest anyway or
% at least I haven't met one who cared to invest in it. If they run into \MATHML\
% they just expect it to be supported.
%
% This variant is adapted to \LMTX\ and might evolve with upcoming variations of
% the standard. As with \SVG\ and probably other \XML\ definitions that target a
% specific domain, \MATHML\ is now more just something for browsers and \HTML\
% usage (although support fluctuates). It's getting less generic (or at least we
% see no more structural elements showing up) and with \CSS\ becoming the
% recomended way to control local optimizations it indeed looks like only browsers
% are the target, unless we can just ignore all that. Limitations of renderers,
% specific usage, application determined \XML\ ... where will it end. All we can do
% is follow and try to keep up, but maybe the more web it becomes, the less \TEX\
% users need to care (read: can stick to \TEX\ coding).
%
% See the \MKII\ and \MKIV\ files for a history and the hacks we had to use in
% the past.

\writestatus{loading}{ConTeXt XML Macros / MathML Renderer}

\unprotect

\usemodule[x][calcmath]
%usemodule[x][asciimath]

%\startmodule [mathml]

\registerctxluafile{ulip-mathml}{autosuffix}

\setupxml[\c!entities=\v!yes] % load big entities table

\def\ctxmodulemathml#1{\ctxlua{moduledata.mathml.#1}}

\startxmlsetups xml:mml:define
    \xmlsetsetup{#1} {(formula|subformula)} {mml:formula}
    \xmlfilter  {#1} {omt:*/function(remapopenmath)}
    \xmlfilter  {#1} {mml:bind/function(remapmmlbind)}
    \xmlfilter  {#1} {mml:csymbol/function(remapmmlcsymbol)}
    \xmlsetsetup{#1} {mml:*} {mml:*}
    \xmlsetsetup{#1} {mml:apply/mml:apply/mml:inverse/../..} {mml:apply:inverse}
    \xmlstrip   {#1} {(mml:mi|mml:mo|mml:mn|mml:csymbol)}
\stopxmlsetups

\xmlregisterns{omt}{openmath}
\xmlregisterns{mml}{mathml}

\xmlregistersetup{xml:mml:define}

\xmlmapvalue {mml:math:mode}    {display} {\displaymathematics} % we had this already
\xmlmapvalue {mml:math:mode}    {inline}  {\inlinemathematics }

\xmlmapvalue {mml:math:display} {block}   {\displaymathematics} % before this showed up
\xmlmapvalue {mml:math:display} {inline}  {\inlinemathematics }

\xmlmapvalue {mml:math:dir}     {ltr}     {\setfalse\c_math_right_to_left\math_basics_synchronize_direction}
\xmlmapvalue {mml:math:dir}     {rtl}     {\settrue \c_math_right_to_left\math_basics_synchronize_direction}

\edef\mmlconstantone  {1}
\edef\mmlconstantfalse{false}

\startxmlsetups mml:math
    \begingroup
        \enableautofences
        \enableautofencemode
        \xmlval {mml:math:dir} {\xmlatt{#1}{dir}} {}
        \xmlval {mml:math:display} {\xmlatt{#1}{display}} {
            \xmlval {mml:math:mode} {\xmlatt{#1}{mode}} {
                \automathematics
            }
        }
        {
            \xmlflush{#1}
        }
    \endgroup
\stopxmlsetups

\startxmlsetups mml:imath
    \inlinemathematics {
        \enableautofences
        \enableautofencemode
        \xmlflush{#1}
    }
\stopxmlsetups

\startxmlsetups mml:dmath
    \displaymathematics {
        \enableautofences
        \enableautofencemode
        \xmlflush{#1}
    }
\stopxmlsetups

%D First we define some general formula elements.

\startxmlsetups mml:formula
    \edef\mmlformulaid      {\xmlatt{#1}{id}}
    \edef\mmlformulalabel   {\xmlatt{#1}{label}\mmlformulaid}
    \edef\mmlformulasublabel{\xmlatt{#1}{sublabel}\mmlformulaid}
    \doifsomething\mmlformulalabel{\placeformula[\mmlformulalabel]{\mmlformulasublabel}}
    \startformula
        \xmlfirst{#1}{/mml:math}
    \stopformula
\stopxmlsetups


\setfalse\mmlignoredelimiter % alternatively we could turn it on/off inside the start/stop and ignore \left\right\middle otherwise

\let\mmlleftdelimiter       \autofenceopen
\let\mmlmiddledelimiter     \autofencemiddle
\let\mmlrightdelimiter      \autofenceclose
\let\mmlleftorrightdelimiter\autofenceboth

\def\mmlchar#1{\char#1 } % used in lua code

%D The rendering macros:

\def\MMLrm{\mr}

\def\MMLseparator#1{\removeunwantedspaces{#1}\ignorespaces} % nils space after separator
\def\MMLseparator#1{,} % todo, for europe we need to block the space

%D Since I only had the draft of MathML 2 and later 3 as example of rendering, there
%D are probably a lot of omissions and misinterpretations. At least I learned some
%D bits and pieces of math rendering.
%D
%D The main complications were not so much the math, but to find the most efficient
%D way to handle elements without spacing beging messed up. The first implementation
%D was aimed at getting reasonable output, this second implementation is already
%D better in terms of handling nesting, and I will definitely need a third one that
%D has more efficient and less ugly code.
%D
%D The \TEX\ part is not that complicated and once the preprocessor was okay, the
%D rest way just a lot of keying and testing. It all comes down to gobbling,
%D redefining, and not so much to parsing.
%D
%D The second implementation expanded the whole math sequence into an internal \TEX\
%D representation. This is a rather clean and fast process. Filtering and testing
%D takes place by redefining the internal representation macros.
%D
%D The third implementation may look a bit more messy in some respects. This is
%D because in \TEX\ it's not that trivial to implement a tree handler. We use a
%D stack for the \type {apply} element and other sequential content. Occasionally we
%D need to peek into child elements which involves messy code. This implementation
%D is closer to the normal \XML\ handling in \CONTEXT.
%D
%D This fourth variant just improves on the third. It uses some tricks available in
%D \LUAMETATEX\ where we have extended the math machinery.

%D We start with the parent elements and the option handler.

\protected\def\xmlmathmldirective#1{\dosetvalue{MML#1}}

\xmlinstalldirective{mathml}{xmlmathmldirective}

%D In the styles, options can be set with:

\protected\def\setupMMLappearance[#1]{\dodoubleargument\getparameters[MML#1]} % no @@ because passed to lua

%D We will apply inner math to all bits and pieces made up by an \type {apply}.

\def\MMLmathinner
  {\ifinner
     \expandafter\firstofoneargument
   \else
     \expandafter\mathinner
   \fi}

%D Auxiliary MathML macros: (to be generalized)

\def\mmlfirst  #1{\xmlelement{#1}{1}} % we can move these inline if needed
\def\mmlsecond #1{\xmlelement{#1}{2}}
\def\mmlthird  #1{\xmlelement{#1}{3}}
\def\mmlprelast#1{\xmlelement{#1}{-2}}
\def\mmllast   #1{\xmlelement{#1}{-1}}

\starttexdefinition doifelsemmlfunction #1
    \xmldoifelse {#1} {/mml:fn} {
        \firstoftwoarguments
    } {
        \xmldoifelse {#1} {/mml:apply/mml:fn} {
            \firstoftwoarguments
        } {
            \xmldoifelse {#1} {/mml:ci[@type=='fn']} {
                \firstoftwoarguments
            } {
                \secondoftwoarguments
            }
        }
    }
\stoptexdefinition

%D A couple of lists:

\convertargument
    mml:times|mml:divide|mml:power|%
    mml:lt|mml:gt|mml:eq|mml:leq|mml:geq|%
    mml:in|mml:inverse|%
    mml:fn|%
    mml:floor|mml:ceiling|%
    mml:mean|%
    mml:selector|%
    mml:abs|mml:int|mml:limit|mml:sum|mml:product|%
    mml:outerproduct|mml:innerproduct|mml:scalarproduct%
\to \MMLcmainresetlist

\convertargument
    mml:sin|mml:arcsin|mml:sinh|mml:arcsinh|%
    mml:cos|mml:arccos|mml:cosh|mml:arccosh|%
    mml:tan|mml:arctan|mml:tanh|mml:arctanh|%
    mml:cot|mml:arccot|mml:coth|mml:arccoth|%
    mml:csc|mml:arccsc|mml:csch|mml:arccsch|%
    mml:sec|mml:arcsec|mml:sech|mml:arcsech|%
    mml:ln|mml:exp|mml:log|%
    mml:abs|mml:int|mml:limit|mml:sum|mml:product|%
    mml:fn%
\to \MMLcfunctionlist

\convertargument
    mml:sin|mml:arcsin|mml:sinh|mml:arcsinh|%
    mml:cos|mml:arccos|mml:cosh|mml:arccosh|%
    mml:tan|mml:arctan|mml:tanh|mml:arctanh|%
    mml:cot|mml:arccot|mml:coth|mml:arccoth|%
    mml:csc|mml:arccsc|mml:csch|mml:arccsch|%
    mml:sec|mml:arcsec|mml:sech|mml:arcsech|%
    mml:ln|mml:exp|mml:log|%
    mml:abs%
\to \MMLcpurefunctionlist

\convertargument
    mml:diff|mml:partialdiff|mml:root%
\to \MMLcconstructlist

%D We use inner and grouping (begin/end and no b/e) else we get problems with
%D 1/2(1+2) and alike. The problem with apply is that we need to take care of
%D several situations, like:
%D
%D \starttyping
%D <apply> <.../> ...
%D <apply> <fn> ...
%D <apply> <apply> <ci> ...
%D <apply> <apply> <fn> <ci> ...
%D \stoptyping
%D
%D Because we translated version 2 of this renderer into version 3 the following
%D definitions may be sub optimal or more complex than actually needed.

%D We will more more to lua ...

% simple version

\newcount\@MMLlevel \def\MMLcreset{\@MMLlevel\zerocount}

\let\MMLctempresetlist\empty \def\setMMLcreset{\edef\MMLctempresetlist}

\let\MMLdoL\donothing
\let\MMLdoR\donothing

\newcount\mmlapplydepth \def\MMLcreset{\mmlapplydepth\zerocount}

\startxmlsetups mml:apply
    \MMLmathinner {
      % \xmldoif {#1} {/(\MMLcmainresetlist\string|\MMLctempresetlist)} {
      %   \MMLcreset
      % }
        \edef\mmlapplyopentoken {\xmlatt{#1}{open}}
        \edef\mmlapplyclosetoken{\xmlatt{#1}{close}}
        \ifcase\mmlapplydepth
        \orelse\ifempty\mmlapplyopentoken
            \def\mmlapplyopentoken {(}
            \def\mmlapplyclosetoken{)}
        \fi
        \advance\mmlapplydepth\plusone
        \begingroup
        \ifempty\mmlapplyopentoken
           \let\MMLdoL\donothing
           \let\MMLdoR\donothing
        \else
           \edef\MMLdoL{\noexpand\left \mmlapplyopentoken }
           \edef\MMLdoR{\noexpand\right\mmlapplyclosetoken}
        \fi
        \let\MMLctempresetlist\empty
        \xmldoifelse {#1} {/mml:apply} {
%             % <apply> <apply> ... </apply> <ci> .. </ci> </apply>
%             \xmldoifelse {#1} {/mml:apply(mml:plus|mml:minus)} {% [a]
%                 % yet incomplete and rather untested
%                 % <apply> <apply> <minus/> <tan/> <cos/> </apply> <ci>x</ci> </apply>
            } {% [b]
%                 \MMLcreset
            }
%             \MMLdoL
%             \mmlfirst{#1}
%             \ifconditional\somepostponedMMLactions
%                 \postponedMMLactions
%             \else
%                 \left(\MMLcreset\mmlsecond{#1}\right)
%             \fi
%             \MMLdoR
%         } {
            \edef\mmlapplyaction{\xmlfilter{#1}{/*/tag()}}
            \doifelsesetups {mml:apply:mml:\mmlapplyaction} {
                \xmlsetup{#1}{mml:apply:mml:\mmlapplyaction}
             } {
%                 \MMLdoL
                \xmlsetup{#1}{mml:\mmlapplyaction}
%                 \MMLdoR
             }
%         }
        \endgroup
        \advance\mmlapplydepth\minusone
      }
\stopxmlsetups

\startxmlsetups mml:apply:mml:apply
    \xmlflush{#1}
    \xmlall{#1}{../[position()>1]}
\stopxmlsetups

\startxmlsetups mml:apply:mml:fn
    \xmldoifelse {#1} {/mml:fn/mml:ci} {
        \edef\mmlfnci{\xmlstripped{#1}{/mml:fn/mml:ci}}
        \doifelsesetups{mmc:fn:\mmlfnci} {
            \xmlsetup{#1}{mmc:fn:\mmlfnci}
        } {
            \MMLcreset
            \MMLdoL
            \mmlfirst{#1}
            \ifnum\xmlcount{#1}{/*}>\plusone
                \negthinspace % not enough
                \left(\MMLcreset\xmlconcatrange{#1}{/*}{2}{}{\MMLseparator,}\right)
            \fi
            \MMLdoR
        }
    } {
        \MMLcreset
        \MMLdoL
        \xmlall{#1}{/*}
        \MMLdoR
    }
\stopxmlsetups

\startxmlsetups mml:apply:mml:csymbol
    \xmlsetup{#1}{mml:csymbol}% \MMLdoL/MMLdoR to be handled in plugin
\stopxmlsetups

\startxmlsetups mml:apply:mml:ci
    \xmlfirst{#1}{/mml:ci}
    \ifnum\xmlcount{#1}{/*}>\plusone
       \left(\MMLcreset\xmlconcatrange{#1}{/*}{2}{}{\MMLseparator,}\right)
    \fi
\stopxmlsetups

% reln

\startxmlsetups mml:reln
    \writestatus{XML}{MathML element "reln" is obsolete}
\stopxmlsetups

% fn

% plusminus ±

\startxmlsetups mmc:fn:\utfchar{"00B1}
    \MMLdoL
    \xmlconcat{#1}{/[position()>1]}{\utfchar{"00B1}}
    \MMLdoR
\stopxmlsetups

% minusplus

\startxmlsetups mmc:fn:\utfchar{"2213}
    \MMLdoL
    \xmlconcat{#1}{/[position()>1]}{\utfchar{"2213}}
    \MMLdoR
\stopxmlsetups

\startxmlsetups mmc:fn
    \begingroup
        \edef\mmlnoffn{\xmlcount{#1}{/*}}
        \ifnum\mmlnoffn>\plustwo
          \def\MMCfnleft {\left(}
          \def\MMCfnright{\right)}
        \else
          \let\MMCfnleft \relax
          \let\MMCfnright\relax
        \fi
        \xmldoifelse {#1} {/mml:ci} {
            \edef\mmlfnci{\xmltext{#1}{/mml:ci}}
            \doifelsesetups{mmc:fn:\mmlfnci} {
                \xmlsetup{#1}{mmc:fn:\mmlfnci}
            } {
                \MMLcreset
                \mmlfirst{#1}
            }
        } {
            \xmldoifelse {#1} {/mml:apply} {
                \xmldoifelse {#1} {/(mml:plus\string|mml:minus)} {
                    \left(\mmlfirst{#1}\right)
                } {
                    \mmlfirst{#1}
                }
                \ifnum\mmlnoffn>\plusone
                    \left(\xmlall{#1}{/!mml:apply}\right)
                \fi
            } {
                \MMLcreset
                \negthinspace
                \MMCfnleft
                \ifnum\mmlnoffn=\plustwo,\fi
                \xmlconcat{#1}{/*}{2}{}{\MMLseparator,}
                \MMCfnright
            }
        }
    \endgroup
\stopxmlsetups

\startxmlsetups mmc:fn:apply % where used?
    \xmldoifelse {#1} {/mml:ci} {
        \edef\mmlfnci{\xmltext{#1}{/mml:ci}}
        \doifelsesetups{mmc:fn:\mmlfnci} {
            \xmlsetup{#1}{mmc:fn:\mmlfnci}
        } {
            \MMLcreset
            \mmlfirst{#1}
            \ifnum\xmlcount{#1}{/*}>\plusone
                \negthinspace
                \left(\MMLcreset\xmlconcat{#1}{2}{}{\MMLseparator,}\right)
            \fi
        }
    } {
        \endgroup
        \MMLcreset
        \mmlfirst{#1}
    }
\stopxmlsetups

%D The next definition provide a kind of plug-in mechanism (see the open math
%D extension module).

% http://www.publishers.com/somename
%
% called at the lua end

\starttexdefinition mmlapplycsymbol #1#2#3#4
    % #1=full url, #2=name, #3=encoding, #4=text
    \doifelse {#3} {text} {
        \text{#4}
    } {
        \doifelsesetups {mml:csymbol:#1} {
            % full url
            \fastsetup{mml:csymbol:#1}
        } {
            % somename (fallback)
            \doifelsesetups {mml:csymbol:#2} {
                \fastsetup{mml:csymbol:#2}
            } {
                \xmlval{mmc:cs}{#3}{}% todo
            }
        }
    }
\stoptexdefinition

\startxmlsetups mml:csymbol
    \mathml_csymbol{#1}
\stopxmlsetups

\startxmlsetups mml:csymbol:cdots
    \cdots
\stopxmlsetups

% \startxmlsetups mml:csymbol:<url> \stopxmlsetups

%D Alternative b will convert periods into comma's:

\setupMMLappearance[cn]       [\c!alternative=\v!a]
\setupMMLappearance[polar]    [\c!alternative=\v!a]  % a|b|c
\setupMMLappearance[float]    [\c!symbol=\v!no]      % \v!yes|dot
\setupMMLappearance[enotation][\c!symbol=\v!no]      % \v!yes|dot
\setupMMLappearance[base]     [\c!symbol=\v!numbers] % digits|characters|text|no

\startxmlsetups mml:cs \xmlcommand{#1}{/}{mml:cs:\xmlattdef{#1}{type}{default}} \stopxmlsetups
\startxmlsetups mml:ci \xmlcommand{#1}{/}{mml:ci:\xmlattdef{#1}{type}{default}} \stopxmlsetups
\startxmlsetups mml:cn \xmlcommand{#1}{/}{mml:cn:\xmlattdef{#1}{type}{default}} \stopxmlsetups

% helpers cn / todo: \mn{...}

\startxmlsetups mml:cn:default
    \mathopnolimits{\xmlflush{#1}}
\stopxmlsetups

% helpers ci

\startxmlsetups mml:ci:default
    \xmlflush{#1}
\stopxmlsetups

\startxmlsetups mml:ci:set
    {\blackboard{\xmlflush{#1}}} % todo
\stopxmlsetups

\startxmlsetups mml:ci:vector
    \overrightarrow{\xmlflush{#1}}
\stopxmlsetups

\startxmlsetups mml:ci:matrix
    {\bi\xmlflush{#1}}
\stopxmlsetups

\startxmlsetups mml:ci:function
    \xmlflush{#1}% \negthinspace
\stopxmlsetups

\startxmlsetups mml:ci:fn
    \xmlsetup{#1}{mml:ci:function}
\stopxmlsetups

\startxmlsetups mml:ci:complex-cartesian
    \xmlsetup{#1}{mml:cn:complex}
\stopxmlsetups

\startxmlsetups mml:ci:complex
    \xmlsetup{#1}{mml:cn:complex}
\stopxmlsetups

\startxmlsetups mml:ci:complex-polar
    \xmlsetup{#1}{mml:cn:polar}
\stopxmlsetups

\startxmlsetups mml:ci:polar
    \xmlsetup{#1}{mml:cn:polar}
\stopxmlsetups

% helpers ci

\startxmlsetups mml:cn:default
    \xmlflush{#1}
\stopxmlsetups

\startxmlsetups mml:cn:integer
    \edef\mmlintegerbase{\xmlattdef{#1}{base}{}}
    \ifx\mmlintegerbase\empty
        \xmlflush{#1}
    \else
        \doifelse \MMLbasesymbol \v!no {
            \MMLcCNbasedata{\xmlflush{#1}}
        } {
            \MMLcCNbasedata{\xmlflush{#1}}\normalsubscript{
                \hbox {\startimath
                    \mr
                    \scriptscriptstyle
                    \processaction
                        [\MMLbasesymbol]
                        [\v!characters=>\MMLcCNbasestring BODH,
                               \v!text=>\MMLcCNbasestring{BIN}{OCT}{DEC}{HEX},
                            \s!unknown=>\mmlintegerbase]
                \stopimath}
            }
        }
    \fi
\stopxmlsetups

\def\MMLcCNbasedata#1%
  {\ifnum\mmlintegerbase>10 \relax{\mr#1}\else#1\fi}

\def\MMLcCNbasestring#1#2#3#4%
  {\ifnum\mmlintegerbase= 2 #1\orelse
   \ifnum\mmlintegerbase= 8 #2\orelse
   \ifnum\mmlintegerbase=10 #3\orelse
   \ifnum\mmlintegerbase=16 #4\else
         \mmlintegerbase      \fi}

\startxmlsetups mml:cn:polar
    \xmlsetup{#1}{mml:cn:polar:\MMLpolaralternative}
\stopxmlsetups

\startxmlsetups mml:cn:polar:a
    \mathml_cpolar{#1}
\stopxmlsetups

\startxmlsetups mml:cn:polar:b
    {\mr e}\normalsuperscript{\xmlsnippet{#1}{1}+\xmlsnippet{#1}{3}\thinspace{\mr i}}
\stopxmlsetups

\startxmlsetups mml:cn:polar:c
    \exp\left(\xmlsnippet{#1}{1}+\xmlsnippet{#1}{3}\thinspace{\mr i}\right)
\stopxmlsetups

\startxmlsetups mml:cn:complex-polar
    \xmlsetup{#1}{mml:cn:polar}
\stopxmlsetups

\startxmlsetups mml:cn:complex % todo ( )
    \left(\xmlsnippet{#1}{1} + \xmlsnippet{#1}{3}\thinspace{\mr i}\right)
\stopxmlsetups

\startxmlsetups mml:cn:complex-cartesian
    \xmlsetup{#1}{mml:cn:complex}
\stopxmlsetups

\startxmlsetups mml:cn:float
    \doifelse \MMLfloatsymbol \v!no {
        % make sure that e shows up ok
        \mathopnolimits{\xmlflush{#1}}
    } {
        % we should ignore \entities !
        \edef\mmlfloatstring{\xmlflush{#1}}
        \splitstring\mmlfloatstring\at e\to\first\and\last
        \ifempty\first
            \mmlfloatstring
        \orelse\ifempty\last
            \mmlfloatstring
        \else
            \first
            \doifelse \MMLfloatsymbol {dot} \cdot \times
            10\normalsuperscript{\last}
        \fi
    }
\stopxmlsetups

\startxmlsetups mml:cn:real
    \xmlsetup{#1}{mml:cn:float}
\stopxmlsetups

\startxmlsetups mml:cn:e-notation
    \doifelse \MMLenotationsymbol \v!no {
        \xmlsnippet{#1}{1}
        \unskip\mathopnolimits{e}\ignorespaces
        \xmlsnippet{#1}{3}
    } {
        \xmlsnippet{#1}{1}
        \doifelse \MMLenotationsymbol {dot} \cdot
        \times10\normalsuperscript{\xmlsnippet{#1}{3}}
    }
\stopxmlsetups

\startxmlsetups mml:cn:logical
    \mathopnolimits{\xmlflush{#1}}
\stopxmlsetups

\startxmlsetups mml:cn:rational
    \xmldoifelse {#1} {/mml:sep} {
        \mmlfrac
            {\xmlsnippet{#1}{1}}
            {\xmlsnippet{#1}{3}}
    } {
        \xmlflush{#1}
    }
\stopxmlsetups

% interval

\setupMMLappearance[interval][\c!alternative=\v!a,\c!separator={,}]

% when empty element, then it's an apply

\startxmlsetups mml:interval
    \doifelse {\xmltag{#1}} {apply} {
        % #1 == apply
        \let\mmlintervalfirst \mmlsecond
        \let\mmlintervalsecond\mmlthird
        \xmlsetup{#1}{mml:interval:\xmlattributedef{#1}{/mml:interval}{closure}{closed}}
    } {
        % #1 == interval
        \let\mmlintervalfirst \mmlfirst
        \let\mmlintervalsecond\mmlsecond
        \xmlsetup{#1}{mml:interval:\xmlattdef{#1}{closure}{closed}}
    }
\stopxmlsetups

\startxmlsetups mml:interval:closed
    \left[\mmlintervalfirst{#1}\MMLseparator\MMLintervalseparator\mmlintervalsecond{#1}\right]
\stopxmlsetups

\startxmlsetups mml:interval:open-closed
    \doifelse \MMLintervalalternative \v!b {
        \left<\mmlintervalfirst{#1}\MMLseparator\MMLintervalseparator\mmlintervalsecond{#1}\right]
     } {
        \left(\mmlintervalfirst{#1}\MMLseparator\MMLintervalseparator\mmlintervalsecond{#1}\right]
     }
\stopxmlsetups

\startxmlsetups mml:interval:closed-open
    \doifelse \MMLintervalalternative \v!b {
        \left[\mmlintervalfirst{#1}\MMLseparator\MMLintervalseparator\mmlintervalsecond{#1}\right>
     } {
        \left[\mmlintervalfirst{#1}\MMLseparator\MMLintervalseparator\mmlintervalsecond{#1}\right)
     }
\stopxmlsetups

\startxmlsetups mml:interval:open
    \doifelse \MMLintervalalternative \v!b {
        \left<\mmlintervalfirst{#1}\MMLseparator\MMLintervalseparator\mmlintervalsecond{#1}\right>
     } {
        \left(\mmlintervalfirst{#1}\MMLseparator\MMLintervalseparator\mmlintervalsecond{#1}\right)
     }
\stopxmlsetups

% inverse

\setfalse\xmlinversefunction

\startxmlsetups mml:apply:inverse
    \settrue\xmlinversefunction
    \xmlsetup{#1}{mml:\xmlfilter{#1}{/mml:apply/*[2]/tag()}}
\stopxmlsetups

% condition

% maybe a fast \xmlnonfirst

% instead of the following we could do \xmlcontent{#1}{/mml:bvar} etc

\startxmlsetups mml:bvar     \xmlflush{#1} \stopxmlsetups
\startxmlsetups mml:lowlimit \xmlflush{#1} \stopxmlsetups
\startxmlsetups mml:uplimit  \xmlflush{#1} \stopxmlsetups
\startxmlsetups mml:degree   \xmlflush{#1} \stopxmlsetups
\startxmlsetups mml:logbase  \xmlflush{#1} \stopxmlsetups
\startxmlsetups mml:fn       \xmlflush{#1} \stopxmlsetups

\startxmlsetups mml:condition
%     \xmldoif {#1} {/mml:bvar} {
%         \xmlfirst{#1}{/mml:bvar}\mid
%     }
    \xmlall{#1}{/!(mml:condition\string|mml:bvar)}
\stopxmlsetups

% declare

\setupMMLappearance[declare][\c!state=\v!start]

\startxmlsetups mml:declare
    \doif \MMLdeclarestate \v!start {
        \mathopnolimits{declare}
        \mmlfirst{#1}
        \ifnum\xmlcount{#1}{/*}>\plusone
            \thickspace
            \mathopnolimits{as}
            \thickspace
        \fi
        \mmlsecond{#1}
    }
\stopxmlsetups

% lambda

\setupMMLappearance[lambda][\c!alternative=b]

\startxmlsetups mml:lambda
    \begingroup
    \doifelse \MMLlambdaalternative \v!a {
        \lambda\left(\xmlconcat{#1}{/!mml:lambda}{\MMLseparator,}\right)
    } {
        \ifnum\xmlcount{#1}{/mml:bvar}>\plusone
            \left(\xmlconcat{#1}{/mml:bvar}{\MMLseparator,}\right)
        \else
            \xmlfirst{#1}{/mml:bvar}
        \fi
        \mapsto
        \MMLcreset
        \xmlall{#1}{/!(mml:bvar|mml:lambda)}
    }
    \endgroup
\stopxmlsetups

% compose

\startxmlsetups mml:compose
    \begingroup
    \MMLcreset
%   \let\MMLcCIfunction\firstofoneargument % brrr ? ? ?
    \doifelsemmlfunction {#1} {
       \left(\xmlconcat{#1}{/!mml:compose}{\circ}\right)
    } {
       \xmlconcat{#1}{/!mml:compose}{\circ}
    }
    \endgroup
\stopxmlsetups

\startxmlsetups mml:image
    \mathopnolimits{image} \left( {\mr\xmlfilter{#1}{/!mml:image/tag()}} \right)
\stopxmlsetups

\setupMMLappearance[piece][\c!separator=]

\startxmlsetups mml:piecewise
    \processaction
         [\MMLpieceseparator]
         [    \v!yes=>\def\theMMLpieceseparator{,&},
               \v!no=>\def\theMMLpieceseparator{&},
          \s!default=>\def\theMMLpieceseparator{&},
          \s!unknown=>\def\theMMLpieceseparator{\,\,\hbox{\MMLpieceseparator}\,\,}]
    \cases{\xmlflush{#1}}
\stopxmlsetups

\startxmlsetups mml:piece
    \mmlfirst{#1}\theMMLpieceseparator\mathematics{\mmlsecond{#1}}\crcr
\stopxmlsetups

\startxmlsetups mml:otherwise
%     \xmlflush{#1}\MMLcPIECEseparator&{\mr otherwise}\crcr
    \xmlflush{#1}&{\mr otherwise}\crcr
\stopxmlsetups

% end of piece

\startxmlsetups mml:quotient
    \lfloor\mmlsecond{#1}/\mmlthird{#1}\rfloor
\stopxmlsetups

\startxmlsetups mml:factorial
    \xmlall{#1}{/!factorial}!
\stopxmlsetups

\setupMMLappearance [divide] [\c!level=\!!maxcard,\c!alternative=\v!a]

\newcount\mmldividelevel

\startxmlsetups mml:divide
    \advance\mmldividelevel\plusone
    \doifelse \MMLdividealternative \v!b {
        \mmlsecond{#1}/\mmlthird{#1}
    } {
        \ifnum \mmldividelevel > \MMLdividelevel \relax % threshold
            \mmlsecond{#1}/\mmlthird{#1}
        \else
            \MMLcreset
            \mmlfrac{\MMLcreset\mmlsecond{#1}}{\MMLcreset\mmlthird{#1}}
        \fi
    }
    \advance\mmldividelevel\minusone
\stopxmlsetups

% min max

\startxmlsetups mml:min \mathopnolimits{min} \xmlsetup{#1}{mml:minmax} \stopxmlsetups
\startxmlsetups mml:max \mathopnolimits{max} \xmlsetup{#1}{mml:minmax} \stopxmlsetups

\startxmlsetups mml:minmax
    \xmldoif {#1} {/mml:bvar} {
        {}\normalsubscript{\xmlfirst{#1}{/mml:bvar}}
    }
    \left\{
    \xmlconcat{#1}{/!(mml:bvar\string|mml:max\string|mml:min)}{\MMLseparator,}
    \right\}
\stopxmlsetups

% minus plus

\setupMMLappearance [plus] [\c!alternative=\v!a] % b = no sign -> 3 1/4
\setupMMLappearance [sign] [\c!reduction=\v!yes]

% alternative b -> geen sign

% branch needed, else (a-b) + (c-d) goes wrong
% reset check in case of (-x) + 37
% reset check in case of (-x) + 37

\newcount\mmlpluscounter

\startxmlsetups mml:plus
    \doifelse \MMLsignreduction \v!yes {
        \MMLdoL
        \xmlsetup{#1}{mml:plus:reset}
        \xmlcommand{#1}{/!mml:plus}{mml:plus:body}
        \MMLdoR
    } {
        \ifnum\xmlcount{#1}{/!mml:plus}=\plusone
            +\xmlfirst{#1}{/!mml:plus}
        \else
            \MMLdoL
            \xmlconcat{#1}{/!mml:plus}{+}
            \MMLdoR
        \fi
    }
\stopxmlsetups

\startxmlsetups mml:plus:reset
    \mmlpluscounter\zerocount
\stopxmlsetups

\startxmlsetups mml:plus:body
    \advance\mmlpluscounter\plusone
    \ifnum\mmlpluscounter>\plusone
        \xmldoifelse{#1}{/mml:minus} {
            \ifnum\xmlcount{#1}{/!mml:minus}>\plusone
                +
            \fi
        } {
            \doifelse {\xmlatt{#1}{type}} {rational} {
                % fraction
            } {
                +
            }
        }
    \fi
    \xmldirect{#1}
\stopxmlsetups

\newcount\mmlminuscounter

\startsetups mml:minus
    \doifelse \MMLsignreduction \v!yes {
        \ifnum\xmlcount{#1}{/!mml:minus}=\plusone
            -\xmlfirst{#1}{/!mml:minus}
        \else
            \MMLdoL
            \xmlsetup{#1}{mml:minus:reset}
            \xmlcommand{#1}{/!mml:minus}{mml:minus:body}
            \MMLdoR
        \fi
    } {
        \left( % \MMLdoL
        \ifnum\xmlcount{#1}{/!mml:minus}=\plusone
            -\xmlfirst{#1}{/!mml:minus}
        \else
            \xmlsetup{#1}{mml:minus:reset}
            \xmlcommand{#1}{/!mml:minus}{mml:minus:body}
        \fi
        \right) % \MMLdoR
    }
\stopsetups

\startxmlsetups mml:minus:reset
    \mmlminuscounter\zerocount
\stopxmlsetups

\startxmlsetups mml:minus:body
    % we can also use concat here
    \advance\mmlminuscounter\plusone
    \ifnum\mmlminuscounter>\plusone
        -
    \fi
    \xmldirect{#1}
\stopxmlsetups

% power

\setupMMLappearance[power][\c!reduction=\v!yes]

\let\MMLpowerelement\empty

\startxmlsetups mml:power
    \xmldoifelse {#1} {/mml:apply}  {
        \doifelse \MMLpowerreduction \v!yes {
            \xmldoifelse {#1} {/mml:apply/(\MMLcfunctionlist)} {
                \gdef\MMLpowerelement{\mmlthird{#1}}% postpone, no xdef
                \MMLcreset\mmlsecond{#1}
            } {
                \left(\MMLcreset\mmlsecond{#1}\right)\normalsuperscript{\MMLcreset\mmlthird{#1}}
            }
        } {
            \left(\MMLcreset\mmlsecond{#1}\right)\normalsuperscript{\MMLcreset\mmlthird{#1}}
        }
    } {
        \mmlsecond{#1}\normalsuperscript{\MMLcreset\mmlthird{#1}}
    }
\stopxmlsetups

% rem

\startxmlsetups mml:rem
    \xmlconcat{#1}{/!mml:rem}{\mathopnolimits{mod}}
\stopxmlsetups

\setupMMLappearance [times] [\c!symbol=\v!no,\c!auto=\v!yes] % new, auto catches cn cn cn

\startxmlsetups mml:times
    \setMMLcreset{\MMLcfunctionlist\string|\MMLcconstructlist}%
    \doifelse\MMLtimesauto\v!no {
        \let\MMLtimes@@symbol\MMLtimessymbol
    } {
        \xmldoifelse {#1} {/mml:cn[name(1) == 'mml:cn']} {% name(1) is next one
           \doifelseinset\MMLtimessymbol{\v!yes,\v!no} {
                \let\MMLtimes@@symbol\v!yes
            } {
                \let\MMLtimes@@symbol\MMLtimessymbol
            }
        } {
            \let\MMLtimes@@symbol\MMLtimessymbol
        }
    }
    \doifelse\MMLtimes@@symbol\v!yes {
        \xmlconcat{#1}{/!mml:times}{\times}
    } {
        \doifelse\MMLtimes@@symbol{dot} {
            \xmlconcat{#1}{/!mml:times}{\cdot}
        } {
            \doifelse\MMLtimes@@symbol{times} {
                \xmlconcat{#1}{/!mml:times}{\times}
            } {
                \xmlall{#1}{/!mml:times}
            }
        }
    }
\stopxmlsetups

\setupMMLappearance[root][\c!symbol=\v!yes]

\startxmlsetups mml:root
    \xmldoifelse {#1} {/mml:degree} {
        \root
            \doifnot\MMLrootsymbol\v!no{\MMLcreset\xmltext{#1}{/mml:degree}}
        \of
    } {
        \sqrt
    }
        {\MMLcreset\xmlall{#1}{/!(mml:degree\string|mml:root)}}
\stopxmlsetups

% gcd

\startxmlsetups mml:gcd
    \begingroup
        \gcd\left(\MMLcreset\xmlconcat{#1}{/!mml:gcd}{\MMLseparator,}\right)
    \endgroup
\stopxmlsetups

% and or xor implies, not

\startxmlsetups mml:and          \xmlconcat{#1}{/!mml:and}    {\wedge}               \stopxmlsetups
\startxmlsetups mml:or           \xmlconcat{#1}{/!mml:or}     {\vee}                 \stopxmlsetups
\startxmlsetups mml:xor          \xmlconcat{#1}{/!mml:xor}    {\mathopnolimits{xor}} \stopxmlsetups
\startxmlsetups mml:implies      \xmlconcat{#1}{/!mml:implies}{\Rightarrow}          \stopxmlsetups
\startxmlsetups mml:not     \neg \xmlall   {#1}{/!mml:not}                           \stopxmlsetups

% forall exists

%D We need to shift left below rotated A.

\startxmlsetups mml:forall
    \forall \negthinspace \xmlsetup{#1}{mml:forallexists}
\stopxmlsetups

\startxmlsetups mml:exists
    \exists \xmlsetup{#1}{mml:forallexists}
\stopxmlsetups

\def\mmlforallexistslist{mml:bvar\string|mml:forall\string|mml:exists\string|mml:condition}

\startxmlsetups mml:forallexists
    \normalsubscript{\xmlconcat{#1}{/mml:bvar}{\MMLseparator,}}
    \xmldoifelse {#1} {/mml:condition} {
        \thickspace
        \begingroup
            \xmlfirst{#1}{/mml:condition}
        \endgroup
        \ifcase\xmlcount{#1}{/!(\mmlforallexistslist)}\relax
            % nothing
        \or
            % == snelle volgende
            \left\vert
                \MMLcreset \medspace \xmlconcat{#1}{/!(\mmlforallexistslist)}{}
            \right.
        \else
            % special case
            \left\vert
                \matrix {
                    \xmlconcat{#1}{/!(\mmlforallexistslist)}{\hfill\crcr}
                }
            \right.
        \fi
    } {
        :\xmlfirst{#1}{/!(\mmlforallexistslist)}
    }
\stopxmlsetups

\startxmlsetups mml:abs
    \left\vert \MMLcreset\xmlall{#1}{/!mml:abs} \right\vert
\stopxmlsetups

\startxmlsetups mml:conjugate % watch extra {}
    {\overline{\MMLcreset\xmlall{#1}{/!mml:conjugate}}}
\stopxmlsetups

\startxmlsetups mml:arg
    \mathopnolimits{arg} \left( \MMLcreset\xmlall{#1}{/!mml:arg} \right)
\stopxmlsetups

\startxmlsetups mml:real
    \Re \left( \MMLcreset \xmlall{#1}{/!mml:real} \right)
\stopxmlsetups

\startxmlsetups mml:imaginary
    \Im \ left( \MMLcreset \xmlall{#1}{/!mml:imaginary} \right)
\stopxmlsetups

\startxmlsetups mml:lcm
    \mathopnolimits{lcm} \left( \xmlconcat{#1}{/!mml:lcm}{\MMLseparator,} \right)
\stopxmlsetups

\startxmlsetups mml:floor
    \lfloor \xmlall{#1}{/!mml:floor} \rfloor
\stopxmlsetups

\startxmlsetups mml:ceiling
    \lceiling \xmlall{#1}{/!mml:ceiling} \rceiling
\stopxmlsetups

% relations

% apply attr or eq

\setupMMLappearance[relation][\c!align=\v!no]

\xmlmapvalue {mml:relation} {eq}         {=}
\xmlmapvalue {mml:relation} {neq}        {\neq}
\xmlmapvalue {mml:relation} {gt}         {>}
\xmlmapvalue {mml:relation} {lt}         {<}
\xmlmapvalue {mml:relation} {geq}        {\geq}
\xmlmapvalue {mml:relation} {leq}        {\leq}
\xmlmapvalue {mml:relation} {equivalent} {\equiv}
\xmlmapvalue {mml:relation} {approx}     {\approx}
\xmlmapvalue {mml:relation} {factorof}   {\mid}

\startxmlsetups mml:eq         \xmlsetup{#1}{mml:relation} \stopxmlsetups
\startxmlsetups mml:neq        \xmlsetup{#1}{mml:relation} \stopxmlsetups
\startxmlsetups mml:gt         \xmlsetup{#1}{mml:relation} \stopxmlsetups
\startxmlsetups mml:lt         \xmlsetup{#1}{mml:relation} \stopxmlsetups
\startxmlsetups mml:geq        \xmlsetup{#1}{mml:relation} \stopxmlsetups
\startxmlsetups mml:leq        \xmlsetup{#1}{mml:relation} \stopxmlsetups
\startxmlsetups mml:equivalent \xmlsetup{#1}{mml:relation} \stopxmlsetups
\startxmlsetups mml:approx     \xmlsetup{#1}{mml:relation} \stopxmlsetups
\startxmlsetups mml:factorof   \xmlsetup{#1}{mml:relation} \stopxmlsetups

\startxmlsetups mml:relation
    \edef\mmlapplyaction{\xmlfilter{#1}{/*/tag()}}
    \MMLcreset \xmlsetup{#1}{mml:relation:\xmlattdef{#1}{align}{\MMLrelationalign}}
\stopxmlsetups

\startxmlsetups mml:relation:default
    \xmlconcatrange{#1}{/*}{2}{}{\xmlval{mml:relation}{\mmlapplyaction}{[\mmlapplyaction]}}
\stopxmlsetups
\startxmlsetups mml:relation:last
    \eqalign {
        \xmlconcatrange{#1}{/*}{2}{-2}{&\xmlval{mml:relation}{\mmlapplyaction}{[\mmlapplyaction]}\crcr}
        \mmlprelast{#1}&\xmlval{mml:relation}{\mmlapplyaction}{[\mmlapplyaction]}{}\mmllast{#1}
    }
\stopxmlsetups
\startxmlsetups mml:relation:first
    \eqalign {
        \mmlsecond{#1}\xmlval{mml:relation}{\mmlapplyaction}{[\mmlapplyaction]}{}
        &\xmlconcatrange{#1}{/*}{3}{}{\crcr\xmlval{mml:relation}{\mmlapplyaction}{[\mmlapplyaction]}{}&}
    }
\stopxmlsetups
\startxmlsetups mml:relation:left
    \eqalign {
        \xmlconcatrange{#1}{/*}{2}{}{&\xmlval{mml:relation}{\mmlapplyaction}{[\mmlapplyaction]}\crcr}
    }
\stopxmlsetups
\startxmlsetups mml:relation:right
    \eqalign {
        &\xmlconcatrange{#1}{/*}{2}{}{\crcr\xmlval{mml:relation}{\mmlapplyaction}{[\mmlapplyaction]}{}&}
    }
\stopxmlsetups
\startxmlsetups mml:relation:no
    \xmlsetup{#1}{mml:relation:default}
\stopxmlsetups
\startxmlsetups mml:relation:yes
    \xmlsetup{#1}{mml:relation:left}
\stopxmlsetups

% personal goody:

\edef\MMLcmainresetlist{\MMLcmainresetlist\string|becomes}

\xmlmapvalue {mml:relation} {mml:becomes} {:=}

\startxmlsetups mml:becomes \xmlsetup{#1}{mml:relation} \stopxmlsetups

% calculus and vector calculus

\startxmlsetups mml:domainofapplication
    \xmlall{#1}{/!mml:domainofapplication}
\stopxmlsetups

\setupMMLappearance[int][\c!location=\v!top]

\def\doMMLlimits#1{\doifelsevalue{MML#1\c!location}\v!top\limits\nolimits}

\startxmlsetups mml:int
    \MMLcreset
    \xmldoifelse {#1} {/mml:domainofapplication} {
        \int \doMMLlimits{int}\normalsubscript{\xmlfirst{#1}{/mml:domainofapplication}}\relax
    } {
        \xmldoifelse {#1} {/mml:condition} {
            \int \doMMLlimits{int}\normalsubscript{\xmlfirst{#1}{/mml:condition}}\relax
        } {
            \xmldoifelse {#1} {/mml:lowlimit} {
                \int \doMMLlimits{int}\normalsubscript{\xmlfirst{#1}{/mml:lowlimit}}\normalsuperscript{\xmlfirst{#1}{/mml:uplimit}}
            } {
                % funny, why do we have lowlimit/uplimit then
                \xmldoifelse {#1} {/mml:apply/mml:interval} {
                    \int \doMMLlimits{int}\normalsubscript{\xmlindex{#1}{/mml:apply}{2}}\normalsuperscript{\xmlindex{#1}{/mml:apply}{3}}
                } {
                    \int
                }
            }
        }
    }
    \MMLcreset
    \xmldoifelse {#1} {/mml:apply} {
        \doifelsemmlfunction {#1} { % todo test
            \xmlfirst{#1}{/mml:apply}
        } {
            % if there are too many () now, we need to be more clever
            \left( \xmlfirst{#1}{/mml:apply} \right)
        }
    } {
        \xmlfirst{#1}{/mml:ci}
    }
    \xmldoifelse {#1} {/mml:bvar} {
        \thinspace {\mr d} \xmlfirst{#1}{/mml:bvar}
    } {
        % nothing
    }
\stopxmlsetups

\setupMMLappearance[diff][\c!location=\v!top,\c!alternative=\v!a]

\startxmlsetups mml:diff
   \MMLcreset
   \doifelse \MMLdiffalternative \v!a {
        \xmldoifelse {#1} {/mml:lambda} {
            % a special case (mathadore/openmath)
            \mmlfrac {
                d
                \normalsuperscript
                {\xmlfirst{#1}{/mml:bvar}\xmlfirst{#1}{/mml:cn}}
                {\xmlfirst{#1}{/mml:lambda}\xmlfirst{#1}{/mml:ci}}
            } {
                d
                {\xmlfirst{#1}{/mml:bvar}\xmlfirst{#1}{/mml:ci}}
                \normalsuperscript
                {\xmlfirst{#1}{/mml:bvar}\xmlfirst{#1}{/mml:cn}}
            }
        } {
            \xmldoifelse {#1} {/mml:bvar} {
                \mmlfrac {
                    {\mr d}{
                        \xmldoifelse {#1} {/mml:degree} {
                            \normalsuperscript{\xmlconcat{#1}{/mml:degree}\empty}
                        } {
                            \xmldoif {#1} {/mml:bvar/mml:degree} {
                                \normalsuperscript{\xmlconcat{#1}{/mml:bvar/mml:degree}+}
                            }
                        }
                    }
                    \doif \MMLdifflocation \v!top {
                        \xmldoifelse {#1} {/mml:ci} {
                            \xmlfirst{#1}{/mml:ci}
                        } {
                            \MMLcreset
                            \ifnum\xmlcount{#1}{/mml:apply/*}>\plustwo % hack
                                \left(\xmlfirst{#1}{/mml:apply}\right)
                            \else
                                \xmlfirst{#1}{/mml:apply}
                            \fi
                        }
                    }
                } {
                    {\mr d}
                    \xmlfirst{#1}{/mml:bvar/!mml:degree}
                    \xmldoif {#1} {/mml:bvar/mml:degree} {
                        \normalsuperscript{\xmlfirst{#1}{/mml:bvar/mml:degree}}
                    }
                }
                \doifnot \MMLdifflocation \v!top {
                    \left(\MMLcreset\xmlfirst{#1}{/(mml:apply\string|mml:ci)}\right)
                }
            } {
                % beware, the second {} is needed for the superscript
                \xmlconcatrange{#1}{/*}{2}{}{}\normalsuperscript\prime
            }
        }
    } {
        \MMLcreset
        \xmlfirst{#1}{/(mml:apply\string|mml:ci)}
        % there can be problems with nested diff's: \normalsuperscript\normalsuperscript{} error
        % so we add an empty group here
        {}\normalsuperscript
        {
            \xmldoifelse {#1} {/mml:degree} {
                \edef\mmldegree{\xmlfirst{#1}{/mml:degree/mml:cn}}
                \ifempty\mmldegree
                    % what to do here
                \else
                    \dorecurse\mmldegree\prime
                \fi
            } {
                \prime
            }
        }
    }
\stopxmlsetups

\startxmlsetups mml:partialdiff
    \xmldoifelse {#1} {/mml:list} {
        {\mr D}\normalsubscript{
            \begingroup
                \setfalse\mmllistdelimiters
                \xmlall{#1}{/mml:list}
            \endgroup
        }
        \xmlfirst{#1}{/(mml:apply\string|mml:reln\string|mml:ci\string|mml:cn)}
    } {
        \xmldoifelse {#1} {/mml:bvar} {
            \mmlfrac {
                {\mr d}\normalsuperscript{
                    \xmldoifelse {#1} {/mml:degree} {
                        \xmlconcat{#1}{/mml:degree}\empty
                    } {
                        \xmlconcat{#1}{/mml:bvar/mml:degree}+
                    }
                }
                \MMLcreset
                \xmlfirst{#1}{/(mml:apply\string|mml:reln\string|mml:ci\string|mml:cn)}
            } {
                \xmldoif {#1}{/mml:bvar/!mml:degree} {
                    \xmlfirst{#1}{/mml:bvar/!mml:degree} \,
                }
                {\mr d}\xmlfirst{#1}{/(mml:apply\string|mml:reln\string|mml:ci\string|mml:cn)}
                \xmldoif {#1} {/mml:bvar/mml:degree} {
                    \normalsuperscript{\xmlfirst{#1}{/mml:bvar/mml:degree}}
                }
            }
        } {
            \xmlfirst{#1}{/(mml:apply\string|mml:reln\string|mml:ci\string|mml:cn)}
        }
    }
\stopxmlsetups

\startxmlsetups mml:divergence \mathopnolimits{div}      \xmlall{#1}{/!mml:divergence} \stopxmlsetups
\startxmlsetups mml:grad       \mathopnolimits{grad}     \xmlall{#1}{/!mml:grad}       \stopxmlsetups
\startxmlsetups mml:curl       \mathopnolimits{curl}     \xmlall{#1}{/!mml:curl}       \stopxmlsetups
\startxmlsetups mml:laplacian  \nabla\normalsuperscript2 \xmlall{#1}{/!mml:laplacian}  \stopxmlsetups
\startxmlsetups mml:ident      \mathopnolimits{identity} \xmlall{#1}{/!mml:ident}      \stopxmlsetups

\setupMMLappearance[domain]  [symbol=]
\setupMMLappearance[codomain][symbol=]

\startxmlsetups mml:domain
    \doifelsenothing \MMLdomainsymbol {
        \mathopnolimits{domain}\MMLcreset\xmlall{#1}{/!mml:domain}
    } {
        \MMLdomainsymbol\normalsubscript{\xmlall{#1}{/!mml:domain}}
    }
\stopxmlsetups

\startxmlsetups mml:codomain
    \doifelsenothing \MMLcodomainsymbol {
        \mathopnolimits{codomain}\MMLcreset\xmlall{#1}{/!mml:codomain}
    } {
        \MMLcodomainsymbol\normalsubscript{\xmlall{#1}{/!mml:codomain}}
    }
\stopxmlsetups

% theory of sets

\startxmlsetups mml:set
    \left\{
    \xmldoifelse {#1} {/mml:condition} {
        \xmlfirst{#1}{/mml:bvar}\,\middle\vert\,\xmlfirst{#1}{/mml:condition}
    } {
        \xmlconcat{#1}{/!mml:set}{\MMLseparator,}
    }
    \right\}
    \relax % needed
\stopxmlsetups

\settrue\mmllistdelimiters

\startxmlsetups mml:list
    \begingroup
        \ifconditional\mmllistdelimiters\left [\fi
        \begingroup
            \settrue\mmllistdelimiters
            \xmlconcat{#1}{/!mml:list}{\MMLseparator,}
        \endgroup
        \ifconditional\mmllistdelimiters\right]\fi
    \endgroup
\stopxmlsetups

\startxmlsetups mml:union       \mmlsecond{#1} \cup            \mmlthird{#1} \stopxmlsetups
\startxmlsetups mml:intersect   \mmlsecond{#1} \cap            \mmlthird{#1} \stopxmlsetups
\startxmlsetups mml:in          \mmlsecond{#1} \in             \mmlthird{#1} \stopxmlsetups
\startxmlsetups mml:notin       \mmlsecond{#1} {\not\in}       \mmlthird{#1} \stopxmlsetups
\startxmlsetups mml:subset      \mmlsecond{#1} \subset         \mmlthird{#1} \stopxmlsetups
\startxmlsetups mml:prsubset    \mmlsecond{#1} \subseteq       \mmlthird{#1} \stopxmlsetups
\startxmlsetups mml:notsubset   \mmlsecond{#1} {\not\subset}   \mmlthird{#1} \stopxmlsetups
\startxmlsetups mml:notprsubset \mmlsecond{#1} {\not\subseteq} \mmlthird{#1} \stopxmlsetups
\startxmlsetups mml:setdiff     \mmlsecond{#1} \setminus       \mmlthird{#1} \stopxmlsetups

\startxmlsetups mml:card
    \left\vert \xmlall{#1}{/!mml:card} \right\vert
\stopxmlsetups

\startxmlsetups mml:cartesianproduct
    \xmlconcat{#1}{/!mml:cartesianproduct}{\times}
\stopxmlsetups

% sequences and series

\setupMMLappearance[sum]    [\c!location=\v!top]
\setupMMLappearance[product][\c!location=\v!top]

\xmlmapvalue {mml:sumprod} {sum}     {\sum}
\xmlmapvalue {mml:sumprod} {product} {\prod}

\startxmlsetups mml:sum     \edef\mmlsumprodname{sum}     \xmlsetup{#1}{mml:sumprod} \stopxmlsetups
\startxmlsetups mml:product \edef\mmlsumprodname{product} \xmlsetup{#1}{mml:sumprod} \stopxmlsetups

\def\mmlstackedsubscripts#1%
  {\vbox
     {\baselineskip\zeropoint % hack, taco vragen
      \halign{\startimath\scriptstyle\hss\alignmark\alignmark\hss\stopimath\cr#1\crcr}}}

% unfinished

\startxmlsetups mml:sumprod
    \begingroup
    \xmldoifelse {#1} {/(mml:condition\string|mml:bvar\string|mml:lowlimit)} {
        \def\mmlsumprodlower{
            \normalsubscript{
                \xmldoifelse {#1} {/mml:condition} {
                    \mmlstackedsubscripts{\xmlconcat{#1}{/mml:condition}{\crcr}}
                } {
                    \xmldoif {#1} {/mml:bvar} {
                        \xmlfirst{#1}{/mml:bvar}
                        \xmldoif{#1}{/mml:lowlimit}{=}
                    }
                    \xmlfirst{#1}{/mml:lowlimit}
                }
            }
        }
    } {
        \let\mmlsumprodlower\empty
    }
    \xmldoifelse {#1} {/mml:uplimit} {
        \def\mmlsumprodupper{\normalsuperscript{\xmlfirst{#1}{/mml:uplimit}}}
    } {
        \let\mmlsumprodupper\empty
    }
   \xmldoif {#1} {/mml:interval} { % open math converter gives this
        \edef\mmlintervalfrom{\xmlindex{#1}{/mml:interval}{1}}
        \edef\mmlintervalto  {\xmlindex{#1}{/mml:interval}{2}}
        \ifempty\mmlintervalfrom  \else
            \def\mmlsumprodlower{\normalsubscript{\xmldoif{#1}{/mml:bvar}{\xmlfirst{#1}{/mml:bvar}{=}}\mmlintervalfrom}}
        \fi
        \ifempty\mmlintervalto \else
            \def\mmlsumprodupper{\normalsuperscript{\mmlintervalto}}
        \fi
    }
    \MMLcreset
    \xmlval{mml:sumprod}{\mmlsumprodname}{}\doMMLlimits\mmlsumprodname\mmlsumprodupper\mmlsumprodlower
    \MMLcreset
    \xmldoifelse {#1} {/mml:lambda/mml:apply} {
        \xmlfirst{#1}{/mml:lambda/mml:apply}% a bit of open math conversion mess
    } {
        \xmlfirst{#1}{/(mml:apply\string|mml:lambda\string|mml:ci)}%
    }
    \endgroup
\stopxmlsetups

\setupMMLappearance[limit][\c!location=\v!top]

\startxmlsetups mml:limit
    \MMLcreset \lim
    \doMMLlimits {limit}\normalsubscript{
        \MMLcreset
        \xmldoifelse {#1} {/mml:condition} {
            \xmlfirst{#1}{/mml:condition}
        } {
            \xmldoif {#1} {/mml:bvar} {
                \xmlfirst{#1}{/mml:bvar}\rightarrow
            }
            \xmlfirst{#1}{/mml:lowlimit}
        }
    }
    \begingroup
        % a bit of open math conversion mess, lambda needed for openmath, ok?
        \MMLcreset
        \xmlfirst{#1}{/mml:lambda/mml:apply}
        \xmlfirst{#1}{/(mml:apply\string|mml:lambda)}
    \endgroup
\stopxmlsetups

% consider a faster index

\startxmlsetups mml:tendsto
    \MMLcreset \mmlsecond{#1}
    \xmlval {mml:tendsto:type} {\xmlattdef{#1}{type}{default}} {\rightarrow}
    \MMLcreset \mmlthird{#1}
\stopxmlsetups

\xmlmapvalue {mml:tendsto:type} {above}   {\downarrow}
\xmlmapvalue {mml:tendsto:type} {below}   {\uparrow}
\xmlmapvalue {mml:tendsto:type} {default} {\rightarrow}

% elementary classical functions

\setupMMLappearance[log][\c!location=\v!right]

\startxmlsetups mml:exp
%   {\mr e}\normalsuperscript{\xmlfirst{#1}{/mml:apply\string|mml:reln\string|mml:ci\string|mml:cn}}
    {\mr e}\normalsuperscript{\xmlfirst{#1}{/!mml:exp}}
\stopxmlsetups

\startxmlsetups mml:log
   \xmldoifelse {#1} {/mml:logbase} {
        \doifelse \MMLloglocation \v!left {
            \mathop {
                {}\normalsuperscript{\xmlfirst{#1}{/mml:logbase}}\negthinspace\mathopnolimits{log}
            }
        } {
            \mathopnolimits{log}\normalsubscript{\xmlfirst{#1}{/mml:logbase}}
        }
%         \MMLcreset
        \xmlfirst{#1}{/(mml:apply\string|mml:reln\string|mml:ci\string|mml:cn)}
%         \xmlsetup{#1}{mml:function} % todo, we start elsewhere
%         \mmlthird{#1}
    } {
        \mathopnolimits{log}
%         \MMLcreset
%         \xmlsetup{#1}{mml:function} % todo, we start elsewhere
            \xmlfirst{#1}{/(mml:apply\string|mml:reln\string|mml:ci\string|mml:cn)}
%         \mmlsecond{#1}
    }
\stopxmlsetups

\startxmlsetups mml:ln
    \mathopnolimits{ln}
    \xmlsetup{#1}{mml:function}
\stopxmlsetups

% statistics

\startxmlsetups mml:mean     \overline                             {\mmlsecond{#1}}                          \stopxmlsetups
\startxmlsetups mml:sdev     \sigma                 \left(\MMLcreset\mmlsecond{#1}\right)                    \stopxmlsetups
\startxmlsetups mml:variance \sigma                 \left(\MMLcreset\mmlsecond{#1}\right)\normalsuperscript2 \stopxmlsetups
\startxmlsetups mml:median   \mathopnolimits{median}\left(\MMLcreset\mmlsecond{#1}\right)                    \stopxmlsetups
\startxmlsetups mml:mode     \mathopnolimits{mode}  \left(\MMLcreset\mmlsecond{#1}\right)                    \stopxmlsetups

% moments

\startxmlsetups mml:moment
    \left\langle
        \xmlfirst{#1}{/(mml:apply\string|mml:reln\string|mml:ci\string|mml:cn)}\normalsuperscript{\xmlfirst{#1}{/mml:degree}}
    \right\rangle
    \xmldoif {#1} {mml:momentabout} {
        \normalsubscript{\xmlfirst{#1}{mml:momentabout}}
    }
\stopxmlsetups

% linear algebra

\setupMMLappearance [vector] [\c!direction=\v!horizontal,\c!separator={,}]

\startxmlsetups mml:vector
    \begingroup
    \ifnum\xmlcount{#1}{/*}>\plusone
        \doifelse\MMLvectordirection\v!horizontal {
            \left(\xmlconcat{#1}{/*}{\MMLseparator\MMLvectorseparator}\right)
        } {
            \MMLcreset\left(\matrix{\xmlconcat{#1}{/*}{\MMLseparator\MMLvectorseparator}}\right)
        }
    \else
      \overrightarrow{\charhtstrut\mmlfirst{#1}}
    \fi
    \endgroup
\stopxmlsetups

\settrue\MMCdelmatrix % ( ) when true

\startxmlsetups mml:matrix
    \begingroup
    \MMLcreset
    \ifconditional\MMCdelmatrix
        \left(\matrix{\xmlcommand{#1}{/mml:matrixrow}{mml:matrixrow:do}}\right)
    \else
        \settrue\MMCdelmatrix
        \matrix{\xmlcommand{#1}{/mml:matrixrow}{mml:matrixrow:do}}
    \fi
    \endgroup
\stopxmlsetups

\startxmlsetups mml:matrixrow
    \begingroup
    \MMLcreset
    \left(\xmlsetup{#1}{mml:matrixrow:do}\right)
    \endgroup
\stopxmlsetups

\startxmlsetups mml:matrixrow:do
    \xmlconcat{#1}{/*}{&}\crcr
\stopxmlsetups

\startxmlsetups mml:determinant
    \begingroup
    \setfalse\MMCdelmatrix
    \left|\mmlsecond{#1}\right|
    \endgroup
\stopxmlsetups

\startxmlsetups mml:transpose
    \mmlsecond{#1}\normalsuperscript{\mathopnolimits{T}}
\stopxmlsetups

\startxmlsetups mml:selector
    \MMLmathinner{\mmlsecond{#1}\normalsubscript{\MMLcreset\xmlconcatrange{#1}{/*}{3}{}{\MMLseparator,}}}
\stopxmlsetups

\startxmlsetups mml:vectorproduct \mmlsecond{#1}\times \mmlthird{#1} \stopxmlsetups
\startxmlsetups mml:scalarproduct \mmlsecond{#1}\cdot  \mmlthird{#1} \stopxmlsetups
\startxmlsetups mml:outerproduct  \mmlsecond{#1}\otimes\mmlthird{#1} \stopxmlsetups

% semantic mapping elements

\setupMMLappearance[semantics][\c!state=\v!start]

\startxmlsetups mml:semantics
    \doifelse\MMLsemanticsstate\v!start {
        \xmlall{#1}{/mml:annotation}
    } {
        \xmlall{#1}{/!mml:annotation}
    }
\stopxmlsetups

\startxmlsetups mml:annotation
    \xmldoifelse {#1} {.[oneof(@encoding,'TeX','tex','application/x-tex','TEX','ConTeXt','context','CONTEXT','ctx')]} {
        \xmlflushcontext{#1}
    } {
        \xmldoifelse {#1} {.[oneof(@encoding,'calcmath','cm')]} {
            \expanded{\calcmath{\xmlflush{#1}}}
        } {
            \xmldoifelse {#1} {.[oneof(@encoding,'asciimath','am')]} {
                \ifdefined\asciimath
                    \expanded{\asciimath{\xmlflushpure{#1}}}
                \else
                    \hbox{\tt no am loaded}
                \fi
            } {
                \xmlall{#1}{../!mml:annotation}
            }
        }
    }
\stopxmlsetups

\startxmlsetups mml:annotation-xml
    % maybe diagnostics
\stopxmlsetups

% misc

\startxmlsetups mml:integers       \integers              \stopxmlsetups
\startxmlsetups mml:reals          \reals                 \stopxmlsetups
\startxmlsetups mml:rationals      \rationals             \stopxmlsetups
\startxmlsetups mml:naturalnumbers \naturalnumbers        \stopxmlsetups
\startxmlsetups mml:complexes      \complexes             \stopxmlsetups
\startxmlsetups mml:primes         \primes                \stopxmlsetups
\startxmlsetups mml:exponentiale   \mathopnolimits{e}     \stopxmlsetups
\startxmlsetups mml:imaginaryi     \mathopnolimits{i}     \stopxmlsetups
\startxmlsetups mml:notanumber     \mathopnolimits{NaN}   \stopxmlsetups
\startxmlsetups mml:true           \mathopnolimits{true}  \stopxmlsetups
\startxmlsetups mml:false          \mathopnolimits{false} \stopxmlsetups
\startxmlsetups mml:emptyset       \mathopnolimits{Ø}     \stopxmlsetups
\startxmlsetups mml:pi             \pi                    \stopxmlsetups
\startxmlsetups mml:eulergamma     \gamma                 \stopxmlsetups
\startxmlsetups mml:infinity       \infty                 \stopxmlsetups

% gonio functions

\setupMMLappearance[function][\c!reduction=\v!yes]

% todo: \mfunction which adapts itself when registered as command

% todo: \def\mmlcfunction#1#2{\mathopnolimits{#2}\xmlsetup{#1}{mml:function}}

\startxmlsetups  mml:sin     \mathcommand    {sin}\xmlsetup{#1}{mml:function} \stopxmlsetups
\startxmlsetups  mml:sinh    \mathcommand   {sinh}\xmlsetup{#1}{mml:function} \stopxmlsetups
\startxmlsetups  mml:cos     \mathcommand    {cos}\xmlsetup{#1}{mml:function} \stopxmlsetups
\startxmlsetups  mml:cosh    \mathcommand   {cosh}\xmlsetup{#1}{mml:function} \stopxmlsetups
\startxmlsetups  mml:tan     \mathcommand    {tan}\xmlsetup{#1}{mml:function} \stopxmlsetups
\startxmlsetups  mml:tanh    \mathcommand   {tanh}\xmlsetup{#1}{mml:function} \stopxmlsetups
\startxmlsetups  mml:cot     \mathcommand    {cot}\xmlsetup{#1}{mml:function} \stopxmlsetups
\startxmlsetups  mml:coth    \mathcommand   {coth}\xmlsetup{#1}{mml:function} \stopxmlsetups
\startxmlsetups  mml:csc     \mathcommand    {csc}\xmlsetup{#1}{mml:function} \stopxmlsetups
\startxmlsetups  mml:csch    \mathcommand   {csch}\xmlsetup{#1}{mml:function} \stopxmlsetups
\startxmlsetups  mml:sec     \mathcommand    {sec}\xmlsetup{#1}{mml:function} \stopxmlsetups
\startxmlsetups  mml:sech    \mathcommand   {sech}\xmlsetup{#1}{mml:function} \stopxmlsetups

\startxmlsetups  mml:arcsin  \mathcommand {arcsin}\xmlsetup{#1}{mml:function} \stopxmlsetups
\startxmlsetups  mml:arcsinh \mathcommand{arcsinh}\xmlsetup{#1}{mml:function} \stopxmlsetups
\startxmlsetups  mml:arccos  \mathcommand {arccos}\xmlsetup{#1}{mml:function} \stopxmlsetups
\startxmlsetups  mml:arccosh \mathcommand{arccosh}\xmlsetup{#1}{mml:function} \stopxmlsetups
\startxmlsetups  mml:arctan  \mathcommand {arctan}\xmlsetup{#1}{mml:function} \stopxmlsetups
\startxmlsetups  mml:arctanh \mathcommand{arctanh}\xmlsetup{#1}{mml:function} \stopxmlsetups
\startxmlsetups  mml:arccot  \mathcommand {arccot}\xmlsetup{#1}{mml:function} \stopxmlsetups
\startxmlsetups  mml:arccoth \mathcommand{arccoth}\xmlsetup{#1}{mml:function} \stopxmlsetups
\startxmlsetups  mml:arccsc  \mathcommand {arccsc}\xmlsetup{#1}{mml:function} \stopxmlsetups
\startxmlsetups  mml:arccsch \mathcommand{arccsch}\xmlsetup{#1}{mml:function} \stopxmlsetups
\startxmlsetups  mml:arcsec  \mathcommand {arcsec}\xmlsetup{#1}{mml:function} \stopxmlsetups
\startxmlsetups  mml:arcsech \mathcommand{arcsech}\xmlsetup{#1}{mml:function} \stopxmlsetups

\startxmlsetups mml:function
    \ifempty\MMLpowerelement
        \ifconditional\xmlinversefunction\normalsuperscript{-1}\fi
        \setfalse\xmlinversefunction
    \else
        \normalsuperscript{\ifconditional\xmlinversefunction-\fi\MMLpowerelement}
        \setfalse\xmlinversefunction
        \glet\MMLpowerelement\empty
    \fi
    \xmlsetup{#1}{mml:function:argument}
\stopxmlsetups

\startxmlsetups mml:function:argument
    \doifelse \MMLfunctionreduction \v!yes {
        \xmldoifelse {#1} {/mml:apply} {
            \xmldoifelse {#1} {/mml:apply/(\MMLcfunctionlist\string|mml:divide)}
                \donefalse
                \donetrue
        } {
            \donefalse
        }
    } {
        \donetrue
    }
    % beware, we still flush from 2 up
    \ifdone
        \left(
            \MMLcreset
            \xmlall{#1}{/[position()>1]}% \xmlconcatrange{#1}{/*}{2}{}\empty
        \right)
    \else
        \MMLcreset
        \xmlall{#1}{/[position()>1]}
    \fi
\stopxmlsetups

% PRESENTATION MATHML

% helpers: maybe we can need a setting for the uprights

\xmlmapvalue {mml:s} {normal}                 {\mathupright} % {\mathtf}
\xmlmapvalue {mml:s} {double-struck}          {\mathblackboard}
\xmlmapvalue {mml:s} {italic}                 {\mathit}
\xmlmapvalue {mml:s} {fraktur}                {\mathfraktur}
\xmlmapvalue {mml:s} {script}                 {\mathscript}
\xmlmapvalue {mml:s} {bold}                   {\mb}          % {\mathbf}
\xmlmapvalue {mml:s} {bold-italic}            {\mathbi}
\xmlmapvalue {mml:s} {bold-fraktur}           {\mathfraktur\mathbf}
\xmlmapvalue {mml:s} {bold-script}            {\mathscript\mathbf}
\xmlmapvalue {mml:s} {sans-serif}             {\mathss}
\xmlmapvalue {mml:s} {bold-sans-serif}        {\mathss\mathbf}
\xmlmapvalue {mml:s} {sans-serif-italic}      {\mathss\mathit}
\xmlmapvalue {mml:s} {sans-serif-bold-italic} {\mathss\mathbi}
\xmlmapvalue {mml:s} {monospace}              {\mathtt}

\xmlmapvalue {mml:l} {-}                      {\let\mmlfrac\tfrac}
                                               \let\mmlfrac\frac
\xmlmapvalue {mml:l} {+}                      {\let\mmlfrac\sfrac}

\xmlmapvalue {mml:d} {true}                   {\displaystyle}
\xmlmapvalue {mml:d} {false}                  {\textstyle} % or whatever

\starttexdefinition setmmlmathstyle #1
    \ifxmlattempty{#1}{displaystyle}\else
        \xmlval{mml:d}\xmllastatt\empty
    \fi
    \ifxmlattempty{#1}{mathvariant}\else
        \xmlval{mml:s}\xmllastatt\empty
    \fi
\stoptexdefinition

\starttexdefinition setmmlscriptlevel #1
    \ifxmlattempty{#1}{scriptlevel}
        \let\mmlfrac\frac
    \else
        \xmlval{mml:l}\xmllastatt{\let\mmlfrac\frac}
    \fi
\stoptexdefinition

\starttexdefinition setmmlmathcolor #1
    \ifxmlattempty{#1}{mathcolor}\else
       \directcolor[\xmllastatt]
    \fi
\stoptexdefinition

\starttexdefinition setmmlmathproperties #1
 %  \ifxmlatt{#1}{displaystyle}\empty\else
    \ifxmlattempty{#1}{displaystyle}\else
        \xmlval{mml:d}\xmllastatt\empty
    \fi
    \ifxmlattempty{#1}{mathvariant}\else
        \xmlval{mml:s}\xmllastatt\empty
    \fi
    \ifxmlattempty{#1}{scriptlevel}
        \let\mmlfrac\frac
    \else
        \xmlval{mml:l}\xmllastatt{\let\mmlfrac\frac}
    \fi
    \ifxmlattempty{#1}{mathcolor}\else
       \directcolor[\xmllastatt]
    \fi
\stoptexdefinition

% todo: textbackgrounds / todo: can be combined with new grouping

\starttexdefinition applymmlmathbackground #1#2
    \ifxmlattempty{#1}{mathbackground}
        #2
    \else
        \backgroundline[\xmllastatt]{#2}
    \fi
\stoptexdefinition

\starttexdefinition applymmlsometext #1#2
    \begingroup
        \applymmlmathbackground {#1} {
            \setmmlmathcolor {#1}
            \setmmlmathstyle {#1}
            #2
        }
    \endgroup
\stoptexdefinition

% setups

\startxmlsetups mml:mi % todo: mathsize (unlikely) mathcolor (easy) mathbackground (easy)
    \begingroup
        \pushmathstyle % still needed ?
        \setmmlmathproperties{#1}
        \mathml_mi{#1}
        \popmathstyle % still needed ?
    \endgroup
\stopxmlsetups

\startxmlsetups mml:mn
    \begingroup
        \setmmlmathcolor{#1}
        \mathml_mn{#1}
    \endgroup
\stopxmlsetups

% <m:mo>-</m:mo><m:mn>2</m:mn> and <m:mn>1</m:mn><m:mo>-</m:mo><m:mn>2</m:mn>
%
% spacing between - and 2 is taken care of by tex itself

\startxmlsetups mml:mo
    \begingroup
        \setmmlmathcolor{#1}
        \ifxmlatt{#1}{maxsize}{1}
            \settrue\mmlignoredelimiter
        \orelse\ifxmlatt{#1}{stretchy}{false}
            \settrue\mmlignoredelimiter
        \fi
        \ifxmlattempty{#1}{lspace}\else
            \hskip\xmllastatt\relax % todo: check for dimension
        \fi
        \mathml_mo{#1}
        \ifxmlattempty{#1}{rspace}\else
            \hskip\xmllastatt\relax % todo: check for dimension
        \fi
    \endgroup
\stopxmlsetups

\startxmlsetups mml:mfenced % {} around separator is needed for spacing
    \mathml_mfenced{#1}
\stopxmlsetups

\defineoverlay [mml:enclose:box]                [\useMPgraphic{mml:enclose:box}]
\defineoverlay [mml:enclose:roundedbox]         [\useMPgraphic{mml:enclose:roundedbox}]
\defineoverlay [mml:enclose:circle]             [\useMPgraphic{mml:enclose:circle}]
\defineoverlay [mml:enclose:left]               [\useMPgraphic{mml:enclose:left}]
\defineoverlay [mml:enclose:right]              [\useMPgraphic{mml:enclose:right}]
\defineoverlay [mml:enclose:top]                [\useMPgraphic{mml:enclose:top}]
\defineoverlay [mml:enclose:bottom]             [\useMPgraphic{mml:enclose:bottom}]
\defineoverlay [mml:enclose:updiagonalstrike]   [\useMPgraphic{mml:enclose:updiagonalstrike}]
\defineoverlay [mml:enclose:downdiagonalstrike] [\useMPgraphic{mml:enclose:downdiagonalstrike}]
\defineoverlay [mml:enclose:horizontalstrike]   [\useMPgraphic{mml:enclose:horizontalstrike}]
\defineoverlay [mml:enclose:verticalstrike]     [\useMPgraphic{mml:enclose:verticalstrike}]

\startuseMPgraphic{mml:enclose:box}
    draw OverlayBox withpen pencircle scaled (ExHeight/10) ;
\stopuseMPgraphic
\startuseMPgraphic{mml:enclose:roundedbox}
    draw OverlayBox cornered .5ExHeight withpen pencircle scaled (ExHeight/10) ;
\stopuseMPgraphic
\startuseMPgraphic{mml:enclose:circle}
    draw fullcircle xysized(bbwidth(OverlayBox),bbheight(OverlayBox)) withpen pencircle scaled (ExHeight/10) ;
\stopuseMPgraphic
\startuseMPgraphic{mml:enclose:left}
    draw leftboundary OverlayBox withpen pencircle scaled (ExHeight/10) ;
    setbounds currentpicture to OverlayBox ;
\stopuseMPgraphic
\startuseMPgraphic{mml:enclose:right}
    draw rightboundary OverlayBox withpen pencircle scaled (ExHeight/10) ;
    setbounds currentpicture to OverlayBox ;
\stopuseMPgraphic
\startuseMPgraphic{mml:enclose:top}
    draw topboundary OverlayBox withpen pencircle scaled (ExHeight/10) ;
    setbounds currentpicture to OverlayBox ;
\stopuseMPgraphic
\startuseMPgraphic{mml:enclose:bottom}
    draw bottomboundary OverlayBox withpen pencircle scaled (ExHeight/10) ;
    setbounds currentpicture to OverlayBox ;
\stopuseMPgraphic
\startuseMPgraphic{mml:enclose:updiagonalstrike}
    path p ; p := OverlayBox enlarged -.25ExHeight ;
    draw llcorner p -- urcorner p withpen pencircle scaled (ExHeight/10) ;
    setbounds currentpicture to OverlayBox ;
\stopuseMPgraphic
\startuseMPgraphic{mml:enclose:downdiagonalstrike}
    path p ; p := OverlayBox enlarged -.25ExHeight ;
    draw ulcorner p -- lrcorner p withpen pencircle scaled (ExHeight/10) ;
    setbounds currentpicture to OverlayBox ;
\stopuseMPgraphic
\startuseMPgraphic{mml:enclose:horizontalstrike}
    path p ; p := OverlayBox enlarged -.25ExHeight ;
    draw .5[llcorner p,ulcorner p] -- .5[lrcorner p,urcorner p] withpen pencircle scaled (ExHeight/10) ;
    setbounds currentpicture to OverlayBox ;
\stopuseMPgraphic
\startuseMPgraphic{mml:enclose:verticalstrike}
    path p ; p := OverlayBox enlarged -.25ExHeight ;
    draw .5[llcorner p,lrcorner p] -- .5[ulcorner p,urcorner p] withpen pencircle scaled (ExHeight/10) ;
    setbounds currentpicture to OverlayBox ;
\stopuseMPgraphic

\startxmlsetups mml:menclose
    \edef\mmlmenclosenotation{\mathml_menclosepattern{#1}}
    \ifempty\mmlmenclosenotation
        \xmlflush{#1}
    \else
        \doifelse \mmlmenclosenotation {mml:enclose:longdiv} {
            \overline{\left)\strut\xmlflush{#1}\right.}
        } {
            \doifelse \mmlmenclosenotation {mml:enclose:actuarial} {
                \overline{\left.\strut\xmlflush{#1}\right\vert}
            } {
                \doifelse \mmlmenclosenotation {mml:enclose:radical} {
                    \sqrt{\xmlflush{#1}}
                } {
                    % todo: no framed when longdiv, actuarial or radical ? spec ?
                    \vcenter {
                        \framed
                            [frame=off,strut=no,background={\mmlmenclosenotation}] % offset is kind of undefined
                            {
                             \startpickupmath
                                \expanded{\doifelseinset {mml:enclose:longdiv} {\mmlmenclosenotation}} {
                                    \overline{\left)\strut\xmlflush{#1}\right.}
                                } {
                                    \expanded{\doifelseinset {mml:enclose:actuarial} {\mmlmenclosenotation}} {
                                        \overline{\left.\strut\xmlflush{#1}\right\vert}
                                    } {
                                        \expanded{\doifelseinset {mml:enclose:radical} {\mmlmenclosenotation}} {
                                            \sqrt{\xmlflush{#1}}
                                        } {
                                            \expanded{\doifelseinset {mml:enclose:rule} {\mmlmenclosenotation}} {
                                                \overline{\strut\xmlflush{#1}}
                                            } {
                                                \xmlflush{#1}
                                            }
                                        }
                                    }
                                }
                             \stoppickupmath
                            }
                    }
                }
            }
        }
    \fi
\stopxmlsetups

\xmlmapvalue {mml:mfrac:linethickness} {thin}   {.2pt}
\xmlmapvalue {mml:mfrac:linethickness} {medium} {.4pt}
\xmlmapvalue {mml:mfrac:linethickness} {thick}  {.8pt}
\xmlmapvalue {mml:mfrac:linethickness} {0}      {0pt}

\startxmlsetups mml:mfrac % dodo: handle linethickness in lua + unit
    \begingroup
    \edef\mmlfraclinethickness{\xmlatt{#1}{linethickness}}
    \ifempty\mmlfraclinethickness
        \doifelse{\xmlatt{#1}{bevelled}}{true} {
            \left.\mmlfirst{#1}\middle/\mmlsecond{#1}\right.% \thinspace\middle/\thinspace
        } {
            \mmlfrac{\mmlfirst{#1}}{\mmlsecond{#1}}
        }
    \else % use \ifchknum
        \doifelse {\xmlval{mml:mfrac:linethickness}{\mmlfraclinethickness}{}} {} {
            \scratchdimen\xmlval{mml:mfrac:linethickness}\mmlfraclinethickness{.4pt}
        } {
            % probably not yet ok
            \setdimensionwithunit\scratchdimen\mmlfraclinethickness{pt}
        }
        {
            {\mmlfirst{#1}}
                \above\scratchdimen
            {\mmlsecond{#1}}
        }
    \fi
    \endgroup
\stopxmlsetups

\startxmlsetups mml:ms
    \hbox {
        \tf % else encoding problems
        \ifxmlattempty{#1}{lquote}\symbol[leftquotation]\else\xmllastatt\fi
        \applymmlsometext{#1}{\xmlflush{#1}}
        \ifxmlattempty{#1}{rquote}\symbol[rightquotation]\else\xmllastatt\fi
    }
\stopxmlsetups

\startxmlsetups mml:mstyle
    \begingroup
        \pushmathstyle
        \setmmlmathstyle{#1}
        \setmmlscriptlevel{#1}
        \xmlflush{#1}
        \popmathstyle
    \endgroup
\stopxmlsetups

\setupMMLappearance[text][\c!alternative=\v!b] % a=normal, b=keep spaces

\startxmlsetups mml:mtext
    \text {
        \applymmlsometext{#1}{
            \ifcstok{\MMLtextalternative}\v!a
                \ignorespaces
                \xmlflush{#1}
                \removeunwantedspaces
            \else
                \xmlflush{#1}
            \fi
        }
    }
\stopxmlsetups

\startxmlsetups mml:merror
    \hbox{\startimath\displaystyle\xmlflush{#1}\stopimath}
\stopxmlsetups

\startxmlsetups mml:mphantom
    \phantom{\triggermathstyle\normalmathstyle\ignorespaces\xmlflush{#1}\removeunwantedspaces}
\stopxmlsetups

\startxmlsetups mml:mpadded % todo
    \xmlflush{#1}
\stopxmlsetups

% mrow / option: no fenced

\startxmlsetups mml:maction
    \xmlflush{#1}
\stopxmlsetups

\startxmlsetups mml:mrow
    \begingroup
        \xmlflush{#1}
    \endgroup
\stopxmlsetups

\startxmlsetups mml:msqrt
    \sqrt{\xmlflush{#1}}
\stopxmlsetups

\startxmlsetups mml:mroot
   \root{\mmlsecond{#1}}\of{\mmlfirst{#1}}
\stopxmlsetups

\setupMMLappearance[scripts][\c!alternative=\v!a] % {} rond base

% brrr no { } when limop .. todo: better in lua
% speed up with ifx and setups or just in lua

\let\mmlnucleus\relax

% maybe make helper
% \iftok{\utfmathclass{\xmlraw{#1}{/mml:*[1]}}}{limop}
% \ifcstok{\utfmathclass{\xmlraw{#1}{/mml:*[1]}}}\s!limop

\startxmlsetups mml:msub
    \iftok{\utfmathclass{\xmlraw{#1}{/mml:*[1]}}}{limop}
        \mmlfirst{#1}
        \normalsubscript{\mmlsecond{#1}}
    \orelse\ifcstok{\MMLscriptsalternative}\v!a
        {\mmlfirst{#1}}
        \normalsubscript{\mmlsecond{#1}}
    \else
        \mmlfirst{#1}
        \normalsubscript{\mmlsecond{#1}}
    \fi
\stopxmlsetups

\startxmlsetups mml:msup
    \iftok{\utfmathclass{\xmlraw{#1}{/mml:*[1]}}}{limop}
        \mmlfirst{#1}
        \normalsuperscript{\mmlsecond{#1}}
    \orelse\ifcstok{\MMLscriptsalternative}\v!a
        {\mmlfirst{#1}}
        \normalsuperscript{\mmlsecond{#1}}
    \else
        \mmlfirst{#1}
        \normalsuperscript{\mmlsecond{#1}}
    \fi
\stopxmlsetups

% use mathclass number

\startxmlsetups mml:msubsup
    \iftok{\utfmathclass{\xmlraw{#1}{/mml:*[1]}}}{limop}
        \mmlfirst{#1}
        \normalsubscript{\mmlsecond{#1}}
        \normalsuperscript{\mmlthird{#1}}
    \orelse\ifcstok{\MMLscriptsalternative}\v!a
        {\mmlfirst{#1}}
        \normalsubscript{\mmlsecond{#1}}
        \normalsuperscript{\mmlthird {#1}}
    \else
        \mmlfirst{#1}
        \normalsubscript{\mmlsecond{#1}}
        \normalsuperscript{\mmlthird {#1}}
    \fi
\stopxmlsetups

% helpers

\protected\def\mmlexecutecommand#1%
  {\ifcsname#1\endcsname
     \expandafter\firstoftwoarguments
   \else
     \expandafter\secondoftwoarguments
   \fi
   \lastnamedcs}

\let\mmlextensible\mathml_extensible

\definemathtriplet [\v!mathematics] [mmlovertriplet]   % or will we use a special instance
\definemathtriplet [\v!mathematics] [mmlundertriplet]  % or will we use a special instance
\definemathtriplet [\v!mathematics] [mmldoubletriplet] % or will we use a special instance

% common to munder/mover/munderover : will become core helper (speed up too)

\starttexdefinition protected mmlfencedfirst #1
    \xmlelement{#1}{1}
\stoptexdefinition
\starttexdefinition protected mmlfencedsecond #1
    \xmlelement{#1}{2}
\stoptexdefinition
\starttexdefinition protected mmlfencedthird #1
    \xmlelement{#1}{3}
\stoptexdefinition

% mover

\starttexdefinition protected mmloverabove #1
    \mmlexecutecommand{\utfmathfiller\mmlovertoken} {\mmlfencedsecond{#1}} \relax
\stoptexdefinition
\starttexdefinition protected mmloverbase #1
    \mmlexecutecommand{\utfmathfiller\mmlbasetoken} {\mmlfencedfirst{#1}} \relax
\stoptexdefinition
\starttexdefinition protected mmloverbasefiller #1
    \mmlexecutecommand{e\utfmathcommandfiller\mmlbasetoken} \relax {\mmlfencedsecond{#1}} {}
\stoptexdefinition
\starttexdefinition protected mmloveraccent #1
    \mmlexecutecommand{\utfmathcommandabove\mmlovertoken} \relax {\mmlfencedfirst{#1}}
\stoptexdefinition

\starttexdefinition protected mmlovertext #1
    \mmlovertriplet {\mmloverbase{#1}} {\mmloverabove{#1}} {}
\stoptexdefinition
\starttexdefinition protected mmloveraccentchecker #1
    \edef\mmlovertoken{\mmlextensible{\xmlraw{#1}{/mml:*[2]}}}% /text()
    \doifelseutfmathabove\mmlovertoken \mmloveraccent \mmlovertext {#1}
\stoptexdefinition

\startxmlsetups mml:mover
    \edef\mmlbasetoken{\mmlextensible{\xmlraw{#1}{/mml:*[1]}}}% /text()
    \doifelseutfmathlimop\mmlbasetoken
      {\mmllimopover{#1}}
      {\doifelseutfmathfiller\mmlbasetoken \mmloverbasefiller \mmloveraccentchecker {#1}}
\stopxmlsetups

\starttexdefinition mmllimopover #1
    \mmlbasetoken\normalsuperscript{\mmlfencedsecond{#1}}
\stoptexdefinition

% munder

\starttexdefinition protected mmlunderbelow #1
    \mmlexecutecommand{\utfmathfiller\mmlundertoken} {\mmlfencedsecond{#1}} \relax
\stoptexdefinition
\starttexdefinition protected mmlunderbase #1
    \mmlexecutecommand{\utfmathfiller\mmlbasetoken} {\mmlfencedfirst{#1}} \relax
\stoptexdefinition
\starttexdefinition protected mmlunderbasefiller #1
    \mmlexecutecommand{e\utfmathcommandfiller\mmlbasetoken} \relax {} {\mmlfencedsecond{#1}}
\stoptexdefinition
\starttexdefinition protected mmlunderaccent #1
    \mmlexecutecommand{\utfmathcommandbelow\mmlundertoken} \relax {\mmlfencedfirst{#1}}
\stoptexdefinition

\starttexdefinition protected mmlundertext #1
    \mmlundertriplet {\mmlunderbase{#1}} {} {\mmlunderbelow{#1}}
\stoptexdefinition
\starttexdefinition protected mmlunderaccentchecker #1
    \edef\mmlundertoken{\mmlextensible{\xmlraw{#1}{/mml:*[2]}}}% /text()
    \doifelseutfmathbelow\mmlundertoken \mmlunderaccent \mmlundertext {#1}
\stoptexdefinition

\startxmlsetups mml:munder
    \edef\mmlbasetoken{\mmlextensible{\xmlraw{#1}{/mml:*[1]}}}% /text()
    \doifelseutfmathlimop\mmlbasetoken
      {\mmllimopunder{#1}}
      {\doifelseutfmathfiller\mmlbasetoken \mmlunderbasefiller \mmlunderaccentchecker {#1}}
\stopxmlsetups

\starttexdefinition mmllimopunder #1
    \mmlbasetoken\normalsubscript{\mmlfencedsecond{#1}}
\stoptexdefinition

% munderover

\starttexdefinition protected mmlunderoveraccentcheckerUO #1
    \edef\mmlundercommand{\utfmathcommandbelow\mmlundertoken}
    \edef\mmlovercommand {\utfmathcommandabove\mmlovertoken}
    \edef\mmlbasecommand {\mmlovercommand\mmlundercommand}
    \ifcsname\mmlbasecommand\endcsname
        \lastnamedcs {\mmlfencedfirst{#1}}
    \orelse\ifcsname\mmlundercommand\endcsname
        \ifcsname\mmlovercommand\endcsname
            \lastnamedcs {\csname\mmlundercommand\endcsname{\mmlfencedfirst{#1}}}
        \else
            \mmldoubletriplet {\csname\mmlundercommand\endcsname{\mmlfencedfirst{#1}}} {\mmlfencedthird{#1}\mmlfencedthird{#1}} {}
        \fi
    \orelse\ifcsname\mmlovercommand\endcsname
        \mmldoubletriplet {\csname\mmlovercommand\endcsname{\mmlfencedfirst{#1}}} {} {\mmlfencedsecond{#1}}
    \else
        \mmlunderoveraccentcheckerTT {#1}
    \fi
\stoptexdefinition
\starttexdefinition protected mmlunderoveraccentcheckerUT #1
    \edef\mmlundercommand{\utfmathcommandbelow\mmlundertoken}
    \edef\mmlbasecommand {\mmlundercommand text}
    \ifcsname\mmlbasecommand\endcsname
        \lastnamedcs {\mmlfencedfirst{#1}} {\mmlfencedthird{#1}}
    \orelse\ifcsname\mmlundercommand\endcsname
        \mmldoubletriplet {\csname\mmlundercommand\endcsname{\mmlfencedfirst{#1}}} {\mmlfencedthird{#1}} {}
    \else
        \mmlunderoveraccentcheckerTT {#1}
    \fi
\stoptexdefinition
\starttexdefinition protected mmlunderoveraccentcheckerOT #1
    \edef\mmlovercommand{\utfmathcommandabove\mmlovertoken}
    \edef\mmlbasecommand{\mmlovercommand text}
    \ifcsname\mmlbasecommand\endcsname
        \lastnamedcs {\mmlfencedfirst{#1}} {\mmlfencedsecond{#1}}
    \orelse\ifcsname\mmlovercommand\endcsname
        \mmldoubletriplet {\csname\mmlovercommand\endcsname{\mmlfencedfirst{#1}}} {} {\mmlfencedsecond{#1}}
    \else
        \mmlunderoveraccentcheckerTT {#1}
    \fi
\stoptexdefinition
\starttexdefinition protected mmlunderoveraccentcheckerTT #1
    \mmldoubletriplet {\mmlfencedfirst{#1}} {\mmlfencedthird{#1}} {\mmlfencedsecond{#1}} \relax
\stoptexdefinition
\starttexdefinition protected mmlunderoveraccentchecker #1
    \edef\mmlundertoken{\mmlextensible{\xmlraw{#1}{/mml:*[2]}}}% /text()
    \edef\mmlovertoken {\mmlextensible{\xmlraw{#1}{/mml:*[3]}}}% /text()
    \doifelseutfmathbelow\mmlundertoken {
        \doifelseutfmathabove\mmlovertoken \mmlunderoveraccentcheckerUO \mmlunderoveraccentcheckerUT {#1}
    } {
        \doifelseutfmathabove\mmlovertoken \mmlunderoveraccentcheckerOT \mmlunderoveraccentcheckerTT {#1}
    }
\stoptexdefinition

\starttexdefinition protected mmlunderoverbasefiller #1
    \mmlexecutecommand{e\utfmathcommandfiller\mmlbasetoken} \relax {\mmlfencedthird{#1}} {\mmlfencedsecond{#1}}
\stoptexdefinition

\startxmlsetups mml:munderover
    \edef\mmlbasetoken{\mmlextensible{\xmlraw{#1}{/mml:*[1]}}}% /text()
    \doifelseutfmathlimop\mmlbasetoken
      {\mmllimopunderover{#1}}
      {\doifelseutfmathfiller\mmlbasetoken \mmlunderoverbasefiller \mmlunderoveraccentchecker {#1}}
\stopxmlsetups

\starttexdefinition mmllimopunderover #1
    \mmlbasetoken\normalsuperscript{\mmlfencedthird{#1}}\normalsubscript{\mmlfencedsecond{#1}}
\stoptexdefinition

% tables (mml:mtable, mml:mtr, mml:mlabledtr, mml:mtd)

\startxmlsetups mml:mtable % some more attributes need to be supported
    \vcenter {
        \hbox {% needed because otherwise positions make the vcenter wide
            \mathml_mtable{#1}
        }
    }
\stopxmlsetups

\startxmlsetups mml:mcolumn
    \vbox{\mathml_mcolumn{#1}}% needs checking
\stopxmlsetups

\def\mmlsetfakewidth#1{\setbox\scratchbox\hbox{#1}\scratchdimen\wd\scratchbox}

\def\mmlmcolumndigitspace     {\mmlsetfakewidth     {0}\kern\scratchdimen}
\def\mmlmcolumndigitrule      {\mmlsetfakewidth     {0}\vrule \s!width \scratchdimen \s!height .2\points \s!depth .2\points\relax}
\def\mmlmcolumnsymbolrule     {\mmlsetfakewidth{\times}\vrule \s!width \scratchdimen \s!height .2\points \s!depth .2\points\relax}
\def\mmlmcolumnpunctuationrule{\mmlsetfakewidth     {.}\vrule \s!width \scratchdimen \s!height .2\points \s!depth .2\points\relax}

\setupMMLappearance[mspace][\c!option=] % \v!test

\startxmlsetups mml:mspace
    \begingroup
        \edef\mmlspacetext{\xmlatt{#1}{spacing}}
        \ifempty\mmlspacetext
            \scratchwidth \xmlattdef{#1}{width} \!!zeropoint % must be string
            \scratchheight\xmlattdef{#1}{height}\!!zeropoint
            \scratchdepth \xmlattdef{#1}{depth} \!!zeropoint
            \ifdim\scratchheight=\zeropoint
                \ifdim\scratchdepth=\zeropoint\else
                    \novrule\s!depth\scratchdepth\s!height\zeropoint\s!width\zeropoint
                \fi
            \else
                \novrule\s!depth\zeropoint\s!height\scratchheight\s!width\zeropoint
            \fi
            \ifdim\scratchwidth=\zeropoint
                % maybe we need to do it anyway
            \orelse\ifx\MMLmspaceoption\v!test
                \hbox to \scratchwidth{\showstruts\strut\hss\lower2\exheight\hbox{\infofont\xmlattdef{#1}{width}}\hss\strut}
            \else
                \hskip\scratchwidth
            \fi
        \orelse\ifx\MMLmspaceoption\v!test
            \hbox{\showstruts\strut\phantom{\triggermathstyle\normalmathstyle\mmlspacetext}\strut}
        \else
            \phantom{\triggermathstyle\normalmathstyle\mmlspacetext}
        \fi
    \endgroup
\stopxmlsetups

% later we can do a better job by manipulating node lists

% \startxmlsetups mml:mline
%     % new, rather undefined, we need to capture a few keywords
%     \edef\mmllinewidth {\xmlatt{#1}{linethickness}}
%     \edef\mmllinetext  {\xmlatt{#1}{spacing}}
%     \edef\mmllinelength{\xmlattdef{#1}{length}\!!zeropoint}
%     \ifempty\mmllinewidth
%         \!!deptha.5\linewidth
%     \else
%         \!!deptha.5\dimexpr\mmllinewidth\relax
%     \fi
%     \!!heighta\!!deptha
%     \ifempty\mmllinetext
%         \ifempty\mmllinelength
%             \!!widtha\zeropoint
%         \else
%             \!!widtha\mmllinelength
%         \fi
%     \else
%         \setbox\scratchbox\hbox{\mathematics{\mathstyle{\mmllinetext}}}% not ok
%         \!!widtha\wd\scratchbox
%     \fi
%     \hbox{\vrule\s!width\!!widtha\s!depth\!!deptha\s!height\!!heighta}
% \stopxmlsetups

\startxmlsetups mml:mglyph % probably never ok (hbox is needed in order to switch to normal font)
    \begingroup
    \edef\mmlglyphfontfamily{\xmlatt   {#1}{fontfamily}}
    \edef\mmlglyphalt       {\xmlattdef{#1}{alt}{unknown}}
    \edef\mmlglyphindex     {\xmlatt   {#1}{index}}
    \ifempty\mmlglyphfontfamily
        \hbox{\tttf[no fontfamily specified for \mmlglyphalt]}
    \orelse\ifempty\mmlglyphindex
        \hbox{\tttf[no index specified for \mmlglyphalt]}
    \else
        \hbox{\getglyph\mmlglyphfontfamily\mmlglyphindex}
    \fi
    \endgroup
\stopxmlsetups

\startxmlsetups mml:maligngroup \stopxmlsetups % will be done when needed
\startxmlsetups mml:malignmark  \stopxmlsetups % will be done when needed

\startxmlsetups mml:none        \stopxmlsetups
\startxmlsetups mml:mprescripts \stopxmlsetups

\startxmlsetups mml:mmultiscripts
    \mathml_mmultiscripts{#1}
\stopxmlsetups

% goodie

\definebuffer[mml]

\permanent\protected\def\stopmml{\xmlprocessbuffer{@mml@}{\thedefinedbuffer{mml}}{}}

%\stopmodule

\protect \endinput

% TODO:
%
% <apply><divide/>
%     <apply><minus/>
%         <apply><minus/><ci>b</ci></apply>
%         <apply><minus/><ci>b</ci></apply>
%         <apply><root/> <ci>a</ci></apply>
%     </apply>
%     <apply><minus/>
%         <apply><minus/><ci>b</ci><ci>b</ci></apply>
%         <apply><minus/><ci>b</ci></apply>
%         <apply><root/> <ci>a</ci></apply>
%     </apply>
% </apply>
